\documentclass[a4paper,14pt]{extarticle}

\usepackage[
    left=2cm,
    right=2cm,
    top=2cm,
    bottom=2cm
    ]{geometry}

\usepackage[T2A]{fontenc}
\usepackage[utf8]{inputenc}
\usepackage[russian,english]{babel}
\usepackage{hyperref}
\usepackage{indentfirst}
\usepackage{graphicx}
\usepackage{float}

\setlength\parindent{1cm}
\renewcommand{\baselinestretch}{1.5}

\begin{document}

\hbadness 1000000 \sloppy

\begin{titlepage}
    \newpage
    \begin{center}
    {\bfseries Национальный исследовательский университет \\
        <<Высшая школа экономики>>}
    \vspace{1cm}
    %САНКТ-ПЕТЕРБУРГСКИЙ \\*
    %ГОСУДАРСТВЕННЫЙ УНИВЕРСИТЕТ \\*
    %\hrulefill
    %\end{center}

    %{КАФЕДРА ЯДЕРНОЙ ФИЗИКИ }
    Лицей
    \vspace{10em}



    %\vspace{2.0em}

    %\begin{center}
    Индивидуальная выпускная работа \\
    \end{center}

    \begin{center}
    %\textsc{\textbf{}}
    Отчет о проекте \linebreak
        \textbf{<<Граф Изучения>>}
    \end{center}

    \vspace{7em}

    \hfill \textit{Выполнил} Чубий Савва Андреевич
    \vspace{6em}
    \vspace{\fill}

    \begin{center}
    Москва 2022
    \end{center}

\end{titlepage}


Я люблю программировать! Несколько лет я посещал курсы по олимпиадному
программированию, сейчас же просто люблю писать различные приложения в
свободное время. Именно по этой причине мне было интересно заняться разработкой
полноценного IT-проекта.

В наше время из-за развития информационных технологий мы сталкиваемся с
огромными объемами знаний. Одни (школьники, студенты и т.д.) эти знания
получают, а другие (например, учителя) их передают. При попытке изучить
некоторую информацию, может возникнуть вопрос: "А с чего начать?". Попробовать
найти параграф на эту тему в учебнике или открыть статью в Интернете? ---
Учебник обычно предназначается для его полного прочтения, и "вырвать" один
параграф бывает затруднительно; а статья в интернете, вероятно, будет ссылаться
на другие статьи или использовать термины, которые могут быть неизвестны
читателю. То есть для того, чтобы успешно изучить некоторую тему, нужно сначала
разобраться с набором других тем, для изучения каждой из которых, требуется
ознакомиться ещё с несколькими темами и т.д.

Рассмотрим на конкретном примере: пусть первоклассник Вася услышал о квадратных
уравнениях. Ему стало интересно, и теперь он хочет разобраться с этой темой.
Во-первых, Васе надо узнать, что вообще такое \textit{уравнение}, во-вторых, мальчику
стоит разобраться с арифметическими действиями: \textit{сложением}, \textit{вычитанием} и
\textit{умножением}, а также возведением в натуральную степень (которая определяется
через умножение), и наконец, первокласснику необходимо разобраться с понятием
\textit{дискриминанта}, для понимания которого нужно изучить тождественные
преобразования и операцию взятия квадратного корня (определяемую через
возведение в натуральную степень, которая в свою очередь определяется через
умножение и т.д.). Таким образом, можно видеть, что для изучения даже казалось
бы простой темы, необходимо предварительное изучение большого числа других тем.

Также не стоит забывать, что подобные проблемы могут возникать не только при
получении знаний, но и при их передаче. Ведь и объяснять учебный материал нужно
в правильном порядке.

Таким образом, программа может помочь любому человеку, желающему получить или
передать какие-либо знания. Однако же целевой аудиторией считаются те, кто
занимается этим активнее всего, то есть:

\begin{itemize}
    \item Среди получающих знания:
    \begin{itemize}
        \item Школьники (особенно учащиеся старшей школы)
        \item Студенты (учащиеся высших учебных заведений)
    \end{itemize}

    \item Среди передающих знания:
    \begin{itemize}
        \item Преподаватели (учителя)
        \item Репетиторы
        \item Авторы учебников
        \item Составители учебных программ
    \end{itemize}
\end{itemize}

Для решения этой проблемы я решил разработать программу, которая предоставляет следующий функционал:

\begin{itemize}
\item Вывести граф изучения данной темы
\item Вывести упорядоченный список изучения тем до данной темы (то есть порядок,
  в котором будет удобно изучать нужные темы)
\item Вывести дополнительную информацию о теме (например, описание, ссылки на информацию о ней)
\item Отметить тему как изученную
\item Добавить тему в список "изучить позже"
\item Импортировать/ экспортировать список изученных тем и список "изучить позже" в/ из файла
\item Создать пакет тем (то есть создать несколько новых графов изучения)
\item Добавить темы/ связи/ зависимые пакеты в пакет тем
\item Импортировать/ экспортировать пакет тем в/ из файла
\item Автоматическая генерация предварительного графа изучения на основе данных некоторых
    сайтов, таких как: Wikipedia, Wikibooks, WikiHow и, возможно, других.
\end{itemize}

Итого, программа предоставляет пользователю возможность легко
манипулировать большим количеством тем, взаимодействуя с наглядной моделью графа,
облегчая тем самым процесс обучения или преподавания.

Во-первых, во время процесса разработки была выявлена невозможность
полноценной генерации графа с соответствующих сайтов (так как всей необходимой
информации на этих сайтах не представлено), вследствие чего было принято
решение сделать генерацию полуавтоматической. Во-вторых, для удобства
пользователей были добавлены возможности изменения языка и размера шрифта
приложения. И наконец, в изначальном варианте возможности взаимодействия с
графом были очень скудны, вследствие чего в финальной версии они были
значительно расширены. Итого, в данный момент программа имеет следующий
функционал (также имеются некоторые другие небольшие функции указывать которые
отдельно не имеет смысла):

\begin{itemize}
    \item Создать/ удалить пакет тем/ граф/ тему
    \item Переименовать пакет/ граф
    \item Вывести/ изменить информацию о теме, а именно:
    \begin{itemize}
        \item Название темы
        \item Пакет, которому принадлежит тема
        \item Находится ли тема в списке "изучить позже"/ "изученные"
        \item Описание темы
        \item Список тем от которых зависит данная
    \end{itemize}
    \item Найти тему(-ы) по
    \begin{itemize}
        \item названию
        \item пакету
        \item наличию в списках "изучить позже" и "изученные"
    \end{itemize}
    \item Найти пакет(-ы)/ граф(-ы) по названию
    \item Вывести упорядоченный список изучения тем до данной темы (то есть порядок,
      в котором будет удобно изучать нужные темы)
    \item Полуавтоматически сгенерировать темы (пакет тем) на основе сайта Wikipedia
    \item Редактировать граф в визуальном режиме:
    \begin{itemize}
        \item Добавить/ удалить тему из графа
        \item Перенести (drag and drop) тему на граф
        \item Переместить тему в графе
        \item Добавить/ удалить зависимость между темами
        \item Скрыть/ показать уже изученные темы
        \item Передвинуться по "холсту" с графом, приблизить или отдалить его
    \end{itemize}
    \item Импортировать/ экспортировать пакет тем/ граф
    \item Изменять размер шрифта/ язык приложения
\end{itemize}

Итого, программа предоставляет пользователю возможность легко
манипулировать большим количеством тем, взаимодействуя с наглядной моделью графа,
облегчая тем самым процесс обучения или преподавания.

Работа над проектом содержала следующие этапы:

\begin{itemize}
    \item Изучение процесса работы с базами данных
    \item Изучение библиотеки Qt
    \item Разработка пользовательских сценариев
    \item Проектирование архитектуры программного продукта
    \item Написание программного кода
    \item Окончательные тестирование и отладка
    \item Подготовка к защите
\end{itemize}

Написан же программный продукт был на языке C++ c использованием библиотеки Qt,
выбранной из-за обширности своего функционала. В качестве базы данных я выбрал
sqlite, так как являясь легкой и быстрой она предоставляет мне весь необходимый
функционал.

Перед началом создания проекта я предполагал возможность возникновения следующих проблем:
\begin{itemize}
    \item Нехватка времени для работы над проектом
    \item Трудности с изучением библиотеки Qt
    \item Трудности, связанные с работой с базой данных
    \item Трудности с налаживанием взаимодействия нескольких пакетов тем
    \item Сложности в работе с парсером для добавления функций импорта и экспорта
\end{itemize}
На самом же деле, по ходу работы из этих проблем я столкнулся только с
нехваткой времени, для решения которой мне пришлось тратить меньше времени на
другие дела (в том числе на сон). К тому же ближе к завершению проекта
ослабевало желание продолжать работу над ним, что ещё больше отдаляло процесс
момент его завершения.

Говоря о приобретенных навыках, для создания программы мне понадобилось изучить
библиотеку Qt и методы работы с базами данных (ни с тем, ни с другим я раньше
не сталкивался). Сам же процесс написания дал мне опыт написания большого
приложения. Так как в будущем я планирую заниматься программированием, все эти
навыки несомненно будут для меня очень полезны.

В дальнейшем мой продукт можно было бы портировать на другие операционные системы,
что расширило бы круг людей которые смогли бы им воспользоваться.

\end{document}
